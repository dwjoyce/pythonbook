% Generated by Sphinx.
\def\sphinxdocclass{report}
\documentclass[letterpaper,10pt,english]{sphinxmanual}
\usepackage[utf8]{inputenc}
\DeclareUnicodeCharacter{00A0}{\nobreakspace}
\usepackage{cmap}
\usepackage[T1]{fontenc}
\usepackage{babel}
\usepackage{times}
\usepackage[Bjarne]{fncychap}
\usepackage{longtable}
\usepackage{sphinx}
\usepackage{multirow}
% Sans font
\usepackage{sans}
\renewcommand{\sfdefault}{\rmdefault}

% Start numbering at zero
\setcounter{chapter}{-1}
\renewcommand\thesection{\thechapter.\the\numexpr\value{section}-1\relax}

% Code block border
% \definecolor{VerbatimColor}{rgb}{0.8, 0.8, 0.8}
% \definecolor{VerbatimBorderColor}{rgb}{0.5, 0.5, 0.5}
\fvset{frame=lines}

\title{Marching up and down the code}
\date{July 08, 2015}
\release{0.0.0}
\author{Matthew Joyce, David Joyce}
\newcommand{\sphinxlogo}{}
\renewcommand{\releasename}{Release}
\makeindex

\makeatletter
\def\PYG@reset{\let\PYG@it=\relax \let\PYG@bf=\relax%
    \let\PYG@ul=\relax \let\PYG@tc=\relax%
    \let\PYG@bc=\relax \let\PYG@ff=\relax}
\def\PYG@tok#1{\csname PYG@tok@#1\endcsname}
\def\PYG@toks#1+{\ifx\relax#1\empty\else%
    \PYG@tok{#1}\expandafter\PYG@toks\fi}
\def\PYG@do#1{\PYG@bc{\PYG@tc{\PYG@ul{%
    \PYG@it{\PYG@bf{\PYG@ff{#1}}}}}}}
\def\PYG#1#2{\PYG@reset\PYG@toks#1+\relax+\PYG@do{#2}}

\expandafter\def\csname PYG@tok@sr\endcsname{\def\PYG@tc##1{\textcolor[rgb]{0.00,0.00,0.00}{##1}}\def\PYG@bc##1{\setlength{\fboxsep}{0pt}\colorbox[rgb]{1.00,0.94,1.00}{\strut ##1}}}
\expandafter\def\csname PYG@tok@s\endcsname{\def\PYG@bc##1{\setlength{\fboxsep}{0pt}\colorbox[rgb]{1.00,0.94,0.94}{\strut ##1}}}
\expandafter\def\csname PYG@tok@gh\endcsname{\let\PYG@bf=\textbf\def\PYG@tc##1{\textcolor[rgb]{0.00,0.00,0.50}{##1}}}
\expandafter\def\csname PYG@tok@ow\endcsname{\let\PYG@bf=\textbf\def\PYG@tc##1{\textcolor[rgb]{0.00,0.00,0.00}{##1}}}
\expandafter\def\csname PYG@tok@gr\endcsname{\def\PYG@tc##1{\textcolor[rgb]{1.00,0.00,0.00}{##1}}}
\expandafter\def\csname PYG@tok@mi\endcsname{\let\PYG@bf=\textbf\def\PYG@tc##1{\textcolor[rgb]{0.00,0.00,0.87}{##1}}}
\expandafter\def\csname PYG@tok@err\endcsname{\def\PYG@tc##1{\textcolor[rgb]{1.00,0.00,0.00}{##1}}\def\PYG@bc##1{\setlength{\fboxsep}{0pt}\colorbox[rgb]{1.00,0.67,0.67}{\strut ##1}}}
\expandafter\def\csname PYG@tok@nt\endcsname{\def\PYG@tc##1{\textcolor[rgb]{0.00,0.47,0.00}{##1}}}
\expandafter\def\csname PYG@tok@na\endcsname{\def\PYG@tc##1{\textcolor[rgb]{0.00,0.00,0.80}{##1}}}
\expandafter\def\csname PYG@tok@nb\endcsname{\def\PYG@tc##1{\textcolor[rgb]{0.00,0.44,0.13}{##1}}}
\expandafter\def\csname PYG@tok@sb\endcsname{\def\PYG@bc##1{\setlength{\fboxsep}{0pt}\colorbox[rgb]{1.00,0.94,0.94}{\strut ##1}}}
\expandafter\def\csname PYG@tok@kp\endcsname{\let\PYG@bf=\textbf\def\PYG@tc##1{\textcolor[rgb]{0.00,0.20,0.53}{##1}}}
\expandafter\def\csname PYG@tok@nf\endcsname{\let\PYG@bf=\textbf\def\PYG@tc##1{\textcolor[rgb]{0.00,0.40,0.73}{##1}}}
\expandafter\def\csname PYG@tok@vc\endcsname{\def\PYG@tc##1{\textcolor[rgb]{0.20,0.40,0.60}{##1}}}
\expandafter\def\csname PYG@tok@kr\endcsname{\let\PYG@bf=\textbf\def\PYG@tc##1{\textcolor[rgb]{0.00,0.53,0.00}{##1}}}
\expandafter\def\csname PYG@tok@kc\endcsname{\let\PYG@bf=\textbf\def\PYG@tc##1{\textcolor[rgb]{0.00,0.53,0.00}{##1}}}
\expandafter\def\csname PYG@tok@il\endcsname{\let\PYG@bf=\textbf\def\PYG@tc##1{\textcolor[rgb]{0.00,0.00,0.87}{##1}}}
\expandafter\def\csname PYG@tok@cp\endcsname{\def\PYG@tc##1{\textcolor[rgb]{0.33,0.47,0.60}{##1}}}
\expandafter\def\csname PYG@tok@kd\endcsname{\let\PYG@bf=\textbf\def\PYG@tc##1{\textcolor[rgb]{0.00,0.53,0.00}{##1}}}
\expandafter\def\csname PYG@tok@w\endcsname{\def\PYG@tc##1{\textcolor[rgb]{0.73,0.73,0.73}{##1}}}
\expandafter\def\csname PYG@tok@nc\endcsname{\let\PYG@bf=\textbf\def\PYG@tc##1{\textcolor[rgb]{0.73,0.00,0.40}{##1}}}
\expandafter\def\csname PYG@tok@ss\endcsname{\def\PYG@tc##1{\textcolor[rgb]{0.67,0.40,0.00}{##1}}}
\expandafter\def\csname PYG@tok@gs\endcsname{\let\PYG@bf=\textbf}
\expandafter\def\csname PYG@tok@s1\endcsname{\def\PYG@bc##1{\setlength{\fboxsep}{0pt}\colorbox[rgb]{1.00,0.94,0.94}{\strut ##1}}}
\expandafter\def\csname PYG@tok@go\endcsname{\def\PYG@tc##1{\textcolor[rgb]{0.53,0.53,0.53}{##1}}}
\expandafter\def\csname PYG@tok@s2\endcsname{\def\PYG@bc##1{\setlength{\fboxsep}{0pt}\colorbox[rgb]{1.00,0.94,0.94}{\strut ##1}}}
\expandafter\def\csname PYG@tok@bp\endcsname{\def\PYG@tc##1{\textcolor[rgb]{0.00,0.44,0.13}{##1}}}
\expandafter\def\csname PYG@tok@cm\endcsname{\def\PYG@tc##1{\textcolor[rgb]{0.53,0.53,0.53}{##1}}}
\expandafter\def\csname PYG@tok@no\endcsname{\let\PYG@bf=\textbf\def\PYG@tc##1{\textcolor[rgb]{0.00,0.20,0.40}{##1}}}
\expandafter\def\csname PYG@tok@si\endcsname{\def\PYG@bc##1{\setlength{\fboxsep}{0pt}\colorbox[rgb]{0.93,0.93,0.93}{\strut ##1}}}
\expandafter\def\csname PYG@tok@mf\endcsname{\let\PYG@bf=\textbf\def\PYG@tc##1{\textcolor[rgb]{0.40,0.00,0.93}{##1}}}
\expandafter\def\csname PYG@tok@gd\endcsname{\def\PYG@tc##1{\textcolor[rgb]{0.63,0.00,0.00}{##1}}}
\expandafter\def\csname PYG@tok@nn\endcsname{\let\PYG@bf=\textbf\def\PYG@tc##1{\textcolor[rgb]{0.05,0.52,0.71}{##1}}}
\expandafter\def\csname PYG@tok@kt\endcsname{\let\PYG@bf=\textbf\def\PYG@tc##1{\textcolor[rgb]{0.20,0.20,0.60}{##1}}}
\expandafter\def\csname PYG@tok@mh\endcsname{\let\PYG@bf=\textbf\def\PYG@tc##1{\textcolor[rgb]{0.00,0.33,0.53}{##1}}}
\expandafter\def\csname PYG@tok@sd\endcsname{\def\PYG@tc##1{\textcolor[rgb]{0.87,0.27,0.13}{##1}}}
\expandafter\def\csname PYG@tok@gi\endcsname{\def\PYG@tc##1{\textcolor[rgb]{0.00,0.63,0.00}{##1}}}
\expandafter\def\csname PYG@tok@c1\endcsname{\def\PYG@tc##1{\textcolor[rgb]{0.53,0.53,0.53}{##1}}}
\expandafter\def\csname PYG@tok@k\endcsname{\let\PYG@bf=\textbf\def\PYG@tc##1{\textcolor[rgb]{0.00,0.53,0.00}{##1}}}
\expandafter\def\csname PYG@tok@nd\endcsname{\let\PYG@bf=\textbf\def\PYG@tc##1{\textcolor[rgb]{0.33,0.33,0.33}{##1}}}
\expandafter\def\csname PYG@tok@vg\endcsname{\let\PYG@bf=\textbf\def\PYG@tc##1{\textcolor[rgb]{0.87,0.47,0.00}{##1}}}
\expandafter\def\csname PYG@tok@kn\endcsname{\let\PYG@bf=\textbf\def\PYG@tc##1{\textcolor[rgb]{0.00,0.53,0.00}{##1}}}
\expandafter\def\csname PYG@tok@vi\endcsname{\def\PYG@tc##1{\textcolor[rgb]{0.20,0.20,0.73}{##1}}}
\expandafter\def\csname PYG@tok@gu\endcsname{\let\PYG@bf=\textbf\def\PYG@tc##1{\textcolor[rgb]{0.50,0.00,0.50}{##1}}}
\expandafter\def\csname PYG@tok@sx\endcsname{\def\PYG@tc##1{\textcolor[rgb]{0.87,0.13,0.00}{##1}}\def\PYG@bc##1{\setlength{\fboxsep}{0pt}\colorbox[rgb]{1.00,0.94,0.94}{\strut ##1}}}
\expandafter\def\csname PYG@tok@m\endcsname{\let\PYG@bf=\textbf\def\PYG@tc##1{\textcolor[rgb]{0.40,0.00,0.93}{##1}}}
\expandafter\def\csname PYG@tok@sc\endcsname{\def\PYG@tc##1{\textcolor[rgb]{0.00,0.27,0.87}{##1}}}
\expandafter\def\csname PYG@tok@sh\endcsname{\def\PYG@bc##1{\setlength{\fboxsep}{0pt}\colorbox[rgb]{1.00,0.94,0.94}{\strut ##1}}}
\expandafter\def\csname PYG@tok@nv\endcsname{\def\PYG@tc##1{\textcolor[rgb]{0.60,0.40,0.20}{##1}}}
\expandafter\def\csname PYG@tok@ge\endcsname{\let\PYG@it=\textit}
\expandafter\def\csname PYG@tok@nl\endcsname{\let\PYG@bf=\textbf\def\PYG@tc##1{\textcolor[rgb]{0.60,0.47,0.00}{##1}}}
\expandafter\def\csname PYG@tok@o\endcsname{\def\PYG@tc##1{\textcolor[rgb]{0.20,0.20,0.20}{##1}}}
\expandafter\def\csname PYG@tok@se\endcsname{\let\PYG@bf=\textbf\def\PYG@tc##1{\textcolor[rgb]{0.40,0.40,0.40}{##1}}\def\PYG@bc##1{\setlength{\fboxsep}{0pt}\colorbox[rgb]{1.00,0.94,0.94}{\strut ##1}}}
\expandafter\def\csname PYG@tok@cs\endcsname{\let\PYG@bf=\textbf\def\PYG@tc##1{\textcolor[rgb]{0.80,0.00,0.00}{##1}}}
\expandafter\def\csname PYG@tok@gt\endcsname{\def\PYG@tc##1{\textcolor[rgb]{0.00,0.27,0.87}{##1}}}
\expandafter\def\csname PYG@tok@c\endcsname{\def\PYG@tc##1{\textcolor[rgb]{0.53,0.53,0.53}{##1}}}
\expandafter\def\csname PYG@tok@gp\endcsname{\let\PYG@bf=\textbf\def\PYG@tc##1{\textcolor[rgb]{0.78,0.36,0.04}{##1}}}
\expandafter\def\csname PYG@tok@mo\endcsname{\let\PYG@bf=\textbf\def\PYG@tc##1{\textcolor[rgb]{0.27,0.00,0.93}{##1}}}
\expandafter\def\csname PYG@tok@ne\endcsname{\let\PYG@bf=\textbf\def\PYG@tc##1{\textcolor[rgb]{1.00,0.00,0.00}{##1}}}
\expandafter\def\csname PYG@tok@ni\endcsname{\let\PYG@bf=\textbf\def\PYG@tc##1{\textcolor[rgb]{0.53,0.00,0.00}{##1}}}

\def\PYGZbs{\char`\\}
\def\PYGZus{\char`\_}
\def\PYGZob{\char`\{}
\def\PYGZcb{\char`\}}
\def\PYGZca{\char`\^}
\def\PYGZam{\char`\&}
\def\PYGZlt{\char`\<}
\def\PYGZgt{\char`\>}
\def\PYGZsh{\char`\#}
\def\PYGZpc{\char`\%}
\def\PYGZdl{\char`\$}
\def\PYGZhy{\char`\-}
\def\PYGZsq{\char`\'}
\def\PYGZdq{\char`\"}
\def\PYGZti{\char`\~}
% for compatibility with earlier versions
\def\PYGZat{@}
\def\PYGZlb{[}
\def\PYGZrb{]}
\makeatother

\renewcommand\PYGZsq{\textquotesingle}

\begin{document}

\maketitle
\tableofcontents
\phantomsection\label{index::doc}



\chapter{Starting with Python's IDLE}
\label{0 - Starting with Python's IDLE:starting-with-python-s-idle}\label{0 - Starting with Python's IDLE::doc}
\begin{notice}{note}{Note:}
Use the force
\end{notice}


\chapter{Python as a Calculator}
\label{1 - Python as a Calculator::doc}\label{1 - Python as a Calculator:python-as-a-calculator}

\section{Addition and subtraction}
\label{1 - Python as a Calculator:addition-and-subtraction}
\begin{notice}{note}{Todo}

Add content
\end{notice}

Some random code:

\begin{Verbatim}[commandchars=\\\{\}]
\PYG{n}{fermat} \PYG{o}{=} \PYG{k}{lambda} \PYG{n}{x}\PYG{p}{,} \PYG{n}{y}\PYG{p}{,} \PYG{n}{z}\PYG{p}{:} \PYG{n}{x} \PYG{o}{*}\PYG{o}{*} \PYG{l+m+mi}{3} \PYG{o}{+} \PYG{n}{y} \PYG{o}{*}\PYG{o}{*} \PYG{l+m+mi}{3} \PYG{o}{==} \PYG{n}{z} \PYG{o}{*}\PYG{o}{*} \PYG{l+m+mi}{3}

\PYG{k+kn}{from} \PYG{n+nn}{engine} \PYG{k+kn}{import} \PYG{n}{RunForrestRun}

\PYG{l+s+sd}{\PYGZdq{}\PYGZdq{}\PYGZdq{}Test code for syntax highlighting!\PYGZdq{}\PYGZdq{}\PYGZdq{}}

\PYG{k}{class} \PYG{n+nc}{Foo}\PYG{p}{:}
        \PYG{k}{def} \PYG{n+nf}{\PYGZus{}\PYGZus{}init\PYGZus{}\PYGZus{}}\PYG{p}{(}\PYG{n+nb+bp}{self}\PYG{p}{,} \PYG{n}{var}\PYG{p}{)}\PYG{p}{:}
                \PYG{n+nb+bp}{self}\PYG{o}{.}\PYG{n}{var} \PYG{o}{=} \PYG{n}{var}
                \PYG{n+nb+bp}{self}\PYG{o}{.}\PYG{n}{run}\PYG{p}{(}\PYG{p}{)}

        \PYG{k}{def} \PYG{n+nf}{run}\PYG{p}{(}\PYG{n+nb+bp}{self}\PYG{p}{)}\PYG{p}{:}
                \PYG{n}{RunForrestRun}\PYG{p}{(}\PYG{p}{)}  \PYG{c}{\PYGZsh{} run along!}
\end{Verbatim}

\begin{notice}{warning}{Warning:}
Never to it this way
\end{notice}


\chapter{Functions and Maths}
\label{2 - Functions and Maths:functions-and-maths}\label{2 - Functions and Maths::doc}

\chapter{Getting help}
\label{3 - Getting help:getting-help}\label{3 - Getting help::doc}

\chapter{Drawing Turtles}
\label{4 - Drawing Turtles:drawing-turtles}\label{4 - Drawing Turtles::doc}

\chapter{Naming your data}
\label{5 - Naming your data:naming-your-data}\label{5 - Naming your data::doc}

\chapter{Asking questions}
\label{6 - Asking questions::doc}\label{6 - Asking questions:asking-questions}

\chapter{Using numbers}
\label{7 - Using numbers::doc}\label{7 - Using numbers:using-numbers}

\chapter{Performing selection}
\label{8 - Performing selection:performing-selection}\label{8 - Performing selection::doc}

\chapter{Decisions, decisions}
\label{9 - Decisions, decisions:decisions-decisions}\label{9 - Decisions, decisions::doc}

\chapter{Combing decisions together}
\label{10 - Combing decisions together:combing-decisions-together}\label{10 - Combing decisions together::doc}

\chapter{Going loopy}
\label{11 - Going loopy:going-loopy}\label{11 - Going loopy::doc}

\chapter{Escaping the cycle}
\label{12 - Escaping the cycle:escaping-the-cycle}\label{12 - Escaping the cycle::doc}

\chapter{Going random}
\label{13 - Going random:going-random}\label{13 - Going random::doc}

\chapter{Grouping data together}
\label{14 - Grouping data together:grouping-data-together}\label{14 - Grouping data together::doc}

\chapter{Slicing sequences}
\label{15 - Slicing sequences:slicing-sequences}\label{15 - Slicing sequences::doc}

\chapter{Walking along data}
\label{16 - Walking along data:walking-along-data}\label{16 - Walking along data::doc}

\chapter{Naming code}
\label{17 - Naming code::doc}\label{17 - Naming code:naming-code}

\chapter{More functions}
\label{18 - More functions:more-functions}\label{18 - More functions::doc}

\chapter{Reading files}
\label{19 - Reading files:reading-files}\label{19 - Reading files::doc}

\chapter{Writing files}
\label{20 - Writing files:writing-files}\label{20 - Writing files::doc}

\chapter{Catching errors}
\label{21 - Catching errors:catching-errors}\label{21 - Catching errors::doc}

\chapter{TODO}
\label{todo:todo}\label{todo::doc}
\begin{notice}{note}{Todo}

Add content
\end{notice}

(The {\hyperref[1 - Python as a Calculator:index-0]{\emph{original entry}}} is located in  /home/matthew/GitHub/pythonbook/1 - Python as a Calculator.rst, line 7.)



\renewcommand{\indexname}{Index}
\printindex
\end{document}
